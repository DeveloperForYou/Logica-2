\chapter{Funzioni Ricorsive}

Lo scopo di questo capitolo \`e quello di dimostrare l'equivalenza tra
i concetti di calcolabilit\`a con registri e calcolabilit\`a con
programmi.  Iniziamo dunque a definire i programmi.

\section{I programmi}
Nella sezione precedente abbiamo visto come il contenuto dei registri,
nelle macchine a registri, possa essere modificato attraverso
determinate \emph{istruzioni} che la macchina \`e in grado di
riconoscere. Una sequenza di queste semplici istruzioni costituisce un
\emph{programma}.

Supponendo $x_i$, $x_j$ e $x_k$ essere nomi di variabili, tali
istruzioni possono essere di quattro tipi:

\begin{enumerate}
\item $x_j \leftarrow 0$ svuota $x_j$ del suo contenuto
\item $x_j \leftarrow x_i $ assegna il contenuto di $x_i$ a $x_j$
\item $x_j \leftarrow x_j+1$ aumenta il contenuto di $x_j$ di uno e
  riassegna il risultato come valore a $x_j$
\item $\textrm{if }x_i \neq0$ then goto $\alpha$ else goto $\beta$ con
  $\alpha$ e $\beta$ istruzioni del programma
\end{enumerate}
	
\begin{nota}
Nella (\emph{4}) \`e possibile sostituire alla condizione $x_i \neq0$
una qualunque formula atomica di un linguaggio che usi le relazioni
$=,\geq,<$ e le operazioni $+,-,successore,\ldots $ e le loro
composizioni.
\end{nota}

\begin{extra}
Si scriva un programma in cui sia presente un'istruzione del tipo
(\emph{4}) avente per condizione $x_i\neq x_j$. (\emph{Suggerimento}:
si pensi ad una funzione in $(x_i,\ldots ,x_j)$ il cui valore sia
diverso da $0$ quando $x_i\neq x_j$)
\end{extra}	

\begin{extra}
Mostrare, con qualche esempio, che \`e possibile sostituire alla
condizione della (\emph{4}) una qualsiasi espressione di
calcolo \newline booleano classico.
\end{extra}	

Per eseguire una \emph{computazione} la macchina deve essere
fornita di un \emph{programma} $P$ e una \emph{configurazione
  iniziale}, che pu\`o essere, ad esempio, una sequenza $a_1, a_2,
...$ di numeri naturali memorizzati nelle variabili $x_i, x_j,
...$. Supponiamo che $P$ sia composto da $s$ istruzioni $I_1, I_2,...,
I_s$.

Allora la macchina inizia la computazione osservando $I_1$, poi $I_2$,
e cos\`i via, a meno che non si incontri un'istruzione di \emph{goto}
che fa saltare la computazione in una delle $s$ istruzioni. La
computazione si ferma quando, e solo quando non c'\`e una prossima
istruzione. Possiamo illustrare questo attraverso un esempio.
	
\begin{esempio} Consideriamo il seguente programma con $s=6$ istruzioni:
\begin{mylisting}
$I_1$: $\textrm{if }x_1 > x_2$ then goto $I_2$ else goto $I_5$
  \\ $I_2$: $x_2 \leftarrow x_2+1$\\ $I_3$: $x_3 \leftarrow
  x_3+1$\\ $I_4$: $\textrm{if }x_1 \neq x_2$ then goto $I_2$ else goto
  $I_5$\\ $I_5$: $x_0 \leftarrow x_3$\\
\end{mylisting}
E con la seguente configurazione iniziale:
\begin{mylisting}
$x_0 = 0, x_1 = 9, x_2 = 7, x_3 = 0, ...$
\end{mylisting}
Durante il calcolo le variabili vengono modificate come segue:
\begin{mylisting}	
$x_0 = 0, x_1 = 9, x_2 = 7, x_3 = 0, ...$ ($I_1: x_1 > x_2$)\\ $x_0 =
  0, x_1 = 9, x_2 = 8, x_3 = 0, ...$ ($I_2$)\\ $x_0 = 0, x_1 = 9, x_2
  = 8, x_3 = 1, ...$ ($I_3$)\\ $x_0 = 0, x_1 = 9, x_2 = 8, x_3 = 1,
  ...$ ($I_4: x_1 \neq x_2$)\\ $x_0 = 0, x_1 = 9, x_2 = 9, x_3 = 1,
  ...$ ($I_2$)\\ $x_0 = 0, x_1 = 9, x_2 = 9, x_3 = 2, ...$
  ($I_3$)\\ $x_0 = 0, x_1 = 9, x_2 = 9, x_3 = 2, ...$ ($I_4: x_1 =
  x_2$)\\ $x_0 = 2, x_1 = 9, x_2 = 9, x_3 = 2, ...$ ($I_5$)\\
\end{mylisting}	
\end{esempio}

\begin{nota}		
Al momento non poniamo la nostra concentrazione su quale funzione
calcola questo programma, ma ci limitiamo ad illustrare in quale modo
opera un programma in senso puramente meccanico.
\end{nota}

\begin{extra}
Dare un programma che memorizza in $x_0$ 1 se $x_1 > x_2$ e 0
altrimenti.
\end{extra}	

Per comprendere il significato del programma e l'andamento della
computazione \`e spesso conveniente desciverlo in modo informale
attraveso un \emph{Flow Chart}, per esempio il flow chart
rappresentante il programma dell'Esempio 1.1 \`e dato in Figura
1. Notare che i test contenuti nei rombi rappresentano le istruzioni
\emph{if then else} (4) che hanno quindi due prosecuzioni in base al
risultato del test; mentre i rettangoli sono le istruzioni di
successore o assegnazione che continuano sempre con la prossima
istruzione.
		
\begin{figure}[h] 
\begin{tikzpicture}[node distance = 1.6 cm, auto]
 % Place nodes
   \node[noblock] (start) {START};
    \node [decision, below of=start] (I1) {$x_1 > x_2$};
    \node [block, below of=I1] (I2) {$x_2 \leftarrow x_2 + 1$};
    \node [block, below of=I2] (I3) {$x_3 \leftarrow x_3 + 1$};
    \node [decision, below of=I3] (I4) {$x_1 \neq x_2$};
    \node [block, below of=I4] (I5) {$x_0 \leftarrow x_3$};
   \node[noblock,below of=I5] (stop) {STOP};

    
    % Draw edges 
   \path [line] (start) -- (I1); \path [line] (I1) --
   node{yes}(I2); \path [line] (I1.east) -- ++(1.0,0) -- ++(0,-1) |-
   node [near start] {no}(I5.east); \path [line] (I2) -- (I3); \path
   [line] (I3) -- (I4); \path [line] (I4.west) -- ++(-.8,0) --
   ++(-.8,0) |- node [near start] {yes}(I2.west); \path [line] (I4)
   --node{no} (I5); \path [line] (I5) -- (stop);
\end{tikzpicture}
\caption{Flow Chart}   
	
\end{figure}
	
Dall'esempio 1 notiamo inoltre come il comando \emph{goto} nella
definizione dell'\emph{if} renda possibile l'esecuzione di cicli,
infatti l'istruzione $I_4$ riporta l'esecuzione all'istruzione $I_2$
tante volte fino a quando la condizione $x_1 \neq x_2$ non viene
soddisfatta.
	
Pi\`u in generale possiamo dire che il comando \emph{goto} pu\`o
essere utilizzato per l'esecuzione di un ciclo \emph{for}.  Infatti
con $n$ fissato il ciclo:\\
$$ \left[
\begin{array}{l}
for \; $i = 1$ \; to \; $n$\\ \ \ \ \ \left[p\;
programma\right. \\ next \; $i$\\
\end{array} \right.
\
$$ 
viene implementato dal programma nell'Esempio 1.2.

\newpage
\begin{esempio}
\end{esempio}
\begin{minipage}[c]{.50\textwidth}
\begin{mylisting}
 $I_1: x_i \leftarrow 0 $\\ $I_2: x_i \leftarrow x_i + 1 $ \\ $I_3:
  \left[p\; programma] \right. $\\ $I_4:$ if $n-x_i\neq 0 $ then \\
\hspace{\stretch{1}} goto $I_2$ else goto stop\\
\end{mylisting}
\end{minipage}
\hspace{5mm}%
\begin{minipage}[c]{.50\textwidth}
\begin{tikzpicture}[node distance = 1.6 cm, auto]
    % Place nodes 
\node[noblock] (start) {START}; \node [block, below
  of=start] (I1) {$x_i \leftarrow 0$}; \node [block, below of=I1] (I2)
{$x_i \leftarrow x_i+ 1$}; \node [block, below of=I2] (I3)
{programma}; \node [decision, below of=I3] (I4) {$n - x_i \neq 0$};
\node[noblock,below of=I4] (stop) {STOP};

    % Draw edges
\path [line] (start) -- (I1);
\path [line] (I1) -- (I2);
\path [line] (I2) -- (I3);
\path [line] (I3) -- (I4);
\path [line] (I4.west) -- ++(-.8,0) -- ++(-.8,0) |- node [near start] {yes}(I2.west);
\path [line] (I4) --node{no} (stop);

\end{tikzpicture}
\end{minipage}
\vspace{5mm}% 
	
Naturalmente, ci possono essere programmi che non terminano mai; per
esempio il semplice programma
	
\begin{mylisting}
$I_1$: if $x_i = x_i $ then goto $I_1$ else goto stop
\end{mylisting}	
non termina mai, e questo deriva dal fatto che la condizione $x_i =
x_i $ \`e sempre vera, quindi la computazione viene rimandata a $I_1$
all'infinito.
	
Ci sono molti modi pi\`u complessi per i quali una computazione
potrebbe eseguire per sempre, ma essenzialmente la causa \`e sempre
una ripetizione o un ciclo come nell'esempio citato sopra.
	
Il problema di decidere se una particolare computazione termina o meno
verr\`a trattato successivamente.\\

\subsection{Funzioni computabili da programmi}
Supponiamo che $f$ sia una funzione da $\mathbb{N}^n$ a $ \mathbb{N}$
con $(n > 1)$; cosa significa dire che $f$ \`e \emph{computabile}
da un programma? \`E naturale pensare in termini di calcolo del valore
$f(a_1, a_2, ... , a_n)$ per mezzo di un programma $P$ con
configurazione iniziale $a_1, a_2, ... , a_n$, Cio\`e, consideriamo
computazione della forma $P(a_1, a_2, ... , a_n)$.

Se tale calcolo termina, c'é un singolo numero che consideriamo come
output o risultato della computazione. Per convenzione poniamo che il
risultato alla fine della computazione sia contenuto nella variabile
$x_0$, mentre il contenuto delle altre variabili pu\`o essere
ingnorato una volta ottenuto il risultato. Diamo dunque la seguente
definizione.

\begin{programmi}
Una funzione $f$ si dice \emph{computabile} da un programma se,
assegnati i valori $a_1,...,a_n$ alle variabili $x_1,...,x_n$, il
programma restituisce il valore $x_0=f(a_1,...,a_n)$.
\end{programmi}

E' facile verificare che con una macchina a registri si possono
eseguire le stesse operazioni viste all'inizio della sezione (e
viceversa), e in effetti vale il seguente teorema.
\begin{progabaco}
Ogni funzione \'e computabile con un programma se e solo se \'e
computabile con una macchina a registri.
\end{progabaco}
\textsc{Dimostrazione} $(\Rightarrow)$ Vediamo come le macchine a
registri sono in grado di calcolare le quattro istruzioni che
costituiscono un programma:\\

\begin{enumerate}
\item 
 $x_j \leftarrow 0$ \hspace{10mm} \xymatrix{ \roundentry{j-}
  \ar@(ul,ul)[] \ar@(ul,dl)[] \ar[r] & stop }
 \vspace{10mm}

\item $x_j \leftarrow x_i $ \hspace{10mm} \xymatrix{ \roundentry{i-}
  \ar@(ul,ul)[] \ar[d] \ar[r] & \roundentry{42-} \ar@(dr,d)[d] \ar[r]
  &stop \\ \roundentry{j+} \ar[d] \ar[u] & \roundentry{i+}
  \ar[u]\\ \roundentry{42+} \ar@(dl,dl)[uu] \\ }
\vspace{10mm}

\item $x_j \leftarrow x_j+1$ \hspace{10mm} \xymatrix{ \roundentry{j+}
  \ar@(ul,ul)[] \ar[d] \\ ...  }
\vspace{10mm}

\item $\textrm{if }x_i \neq0$ then goto $\alpha$ else goto
  $\beta$ \hspace{10mm} \xymatrix{ \roundentry{i-} \ar@(ul,ul)[]
  \ar[d] \ar[r] & \beta\\ \roundentry{i+} \ar[d] \\ \alpha }

\end{enumerate}
	
$(\Leftarrow)$ Si tratta di simulare con (1)$\rightarrow$(4) le
operazioni di una macchina a registri:
\begin{enumerate}
\item esegue l'incremento di uno, la prima operazione elementare di
  una macchina a registri;\\ \\
\hspace{10mm} \xymatrix{\roundentry{j+} \ar@(ul,ul)[] \ar[d] \\ ...}
\hspace{10mm} $x_j \leftarrow x_j+1$

\item la seconda operazione elementare sulle macchine a registri si
  ottiene usando (4) e ponendo in $\alpha$ un'istruzione che sottragga
  1 al registro se non \'e vuoto, altrimenti il programma esegue
  l'istruzione $else$ $\beta$, che corrisponde all'operazione di
  procedere a destra.\\ \\ \xymatrix{ \roundentry{i-} \ar@(ul,ul)[]
    \ar[d] \ar[r] & \beta\\ \alpha }
 \hspace{10mm} $\textrm{if }x_i \neq 0$ then goto $\alpha$ else goto
 $\beta$ \\ con $\alpha = x \stackrel{\centerdot}{-} 1$ e $\beta=$
 'procedi a destra'.
\end{enumerate}

\hspace{\stretch{1}} $\Box$\\
\begin{esempio}[Differenza]
\ Come esempio riportiamo il calcolo della funzione differenza
definita come segue:\\
\begin{center}
$n \stackrel{\centerdot}{-} m= \left\{ \begin{array}{ll} n-m &
    \textrm{se } m \leq n\\ 0 & \textrm{altrimenti}
\end{array} \right.$
\end{center} 

\begin{figure}[h]
\hspace{0cm} \xymatrix{ \roundentry{1-} \ar[r] \ar@(dr,d)[d] & stop
  \\ \roundentry{2-} \ar@(ul,ul)[] \ar[u] \ar[r] & \roundentry{1-}
  \ar[d] \ar[r] & stop \\ & \roundentry{42+} \ar@(d,dl)[u] }
\caption{Macchina a registri che calcola la differenza definita sui
  naturali $n \stackrel{\centerdot}{-} m$}
\end{figure}
	
\ \\ Tale funzione viene calcolata dalla macchina a registri in Figura
2 e dal programma o che abbiamo esaminato nell'Esempio 1.1.
\end{esempio}
	
	
Il teorema appena enunciato stabilisce la completa equivalenza tra i
concetti di calcolabilità con registri e calcolabilità con
programmi. Si noti come attraverso queste equivalenze si giunga ad un
livello di astrazione sempre maggiore, e come proprio questa
progressione ci fornisca strumenti via via pi\'u versatili attraverso
i passaggi per la dimostrazione dei teoremi di incompletezza.
				

\section{Funzioni primitive ricorsive}
Vogliamo definire ora una classe che comprenda tutte e sole le
funzioni calcolabili. Seguiremo a tale scopo Post, il quale si serve
di una definizione ricorsiva. Una definizione \`e detta
\emph{ricorsiva} quando ci\`o che \`e da definirsi viene definito
facendo ricorso a istanze pi\`u elementari dello stesso problema.
Tale metodo consiste nel:
\begin{itemize}
 \item fissare un insieme di funzioni iniziali immediatamente
   calcolabili quale base della procedura di definizione
 \item indicare alcune regole per derivare altre funzioni ricorsive da
   quelle date in partenza (regole che ovviamente preservino la
   calcolabilità delle funzioni derivate)
 \item escludere dalla classe delle funzioni ricorsive quelle funzioni
   che non siano le iniziali o da queste derivabili.
\end{itemize}
Diamo dunque la seguente definizione:

\begin{programmi}
Si dice \emph{funzione primitiva ricorsiva} un elemento della
classe definita induttivamente a partire dalle seguenti funzioni:
\begin{itemize}
\item [a.] la funzione nulla $z$ tale che $z(x)=0$;
\item [b.] la funzione proiezione $p^n_i$ tale che $p_i^n(x_1,\ldots
  ,x_n)=x_i$ $\forall i\leq n$
\item [c.]la funzione successore $s$ tale che $s(x) = x+1$;
\end{itemize}
\end{programmi}
	
Le regole per produrre nuove funzioni sono:
\begin{enumerate}
\item le operazioni di \emph{\emph{composizione}}, che date le
  funzioni primitive ricorsive $f: \mathbb{N}^k \rightarrow
  \mathbb{N}$ e $g_i : \mathbb{N}^n \rightarrow \mathbb{N}$ per $i= 1,
  ... , k$ permette di ottenere una funzione $h : \mathbb{N}^n
  \rightarrow \mathbb{N}$ per cui:\\ $
  h(x_1,.....,x_n)=f(g1(x_1,.....,x_n),....., g_k (x_1,.....,x_n))$
  \\ anch'essa primitiva ricorsiva.
\item lo schema di \emph{\emph{ricorsione primitiva}}, che date
  le funzioni primitive ricorsive $f: \mathbb{N}^k \rightarrow
  \mathbb{N}$ e $g : \mathbb{N}^{k+2} \rightarrow \mathbb{N}$, allora
  \`e primitiva ricorsiva anche la funzione $h : \mathbb{N}^{k+1}
  \rightarrow \mathbb{N}$ che soddisfi il sistema di equazioni:
\begin{center}
$\left\{ \begin{array}{ll} h(x_1,.....,x_n,0)= f(x_1,.....,x_n)
    \\ h(x_1,.....,x_n,s(y))=g(x_1,.....,x_n, y, h(x_1,.....,x_n,y))
				\end{array} \right.$
\end{center}
\end{enumerate}
\vspace{5mm}%
	
La classe delle funzioni primitive ricorsive \`e dunque \emph{chiusa
  rispetto alle operazioni di composizione e ricorsione
  primitiva}. Ora vedremo che questa classe di funzioni, che abbiamo
appena definito in termini matematici, \`e calcolabile dai programmi.

% Ane Santos Herranz

\begin{teorema}
\emph{Ogni funzione primitiva ricorsiva è eseguibile da un programma
(e quindi calcolabile).}\end{teorema}
\begin{proof}
Dobbiamo fare un programma per le tre funzioni di base, per la
composizione generalizzata e la ricorsione. Come sappiamo il output è
sempre in $x_{0}$
\begin{itemize}
\item[a.] Facciamo un programma per la funzione zero:
\begin{mylisting}
$I_1$: $x_{0}\leftarrow0$
\end{mylisting}

\item[b.] Il programma per il successore:\begin{mylisting}$I_1$:
  $x_{0}\leftarrow x_{1}+1$\end{mylisting}

\item[c.] La proiezione:\begin{mylisting}$I_1$: $x_{0}\longleftarrow
  x_{i}$\end{mylisting}
\end{itemize}

\begin{enumerate}
\item La composizione: siano $P, P_1, P_2, \dots, P_n$ programmi per
  le funzioni $f,g_{1},...g_{n}$ rispettivamente. L'output di ogni
  funzione viene salvato in $x_{0}$; quindi, poich\'e vogliamo salvare
  tutti i risultati intermedi del calcolo di $g_{1},\dots,g_{n}$
  dobbiamo copiare di volta in volta $x_0$ in una variabile che siamo
  sicuri che non sia usata da uno dei programmi. Inoltre dobbiamo
  riservarci dello spazio per le variabili di input poich\'e qualche
  programma potrebbe modificarle durante la sua esecuzione.  Per
  sapere dove possiamo salvare tutti questi valori dobbiamo sapere
  quali variabili vengono utilizzate dai programmi.
  Sia quindi
  $$\rho(Q)=max\left\{ i|x_{i}\leftarrow...\in Q\right\}$$
  il massimo indice di una variabile assegnata in $Q$ e
  $$\varepsilon=max\left\{
  \rho(P_{1}),....,\rho(P_{n}),\rho(P),k\right\}$$
  il massimo indice utilizzato nei programmi $P, P_1, \dots, P_n$.

Costruiamo il programma che calcola la composizione come segue:
salviamo l'input, che si trova in $x_{1},\dots,x_{k}$ in
$x_{\varepsilon+1},\dots,x_{\varepsilon+k}$, applichiamo il programma
$P_1$ e copiamo l'output in $x_{\varepsilon+k+1}$:

\begin{mylisting}
$x_{\varepsilon+1}\leftarrow x_{1}$\\
$\vdots$\\
$x_{\varepsilon+k}\leftarrow x_{k}$\\
$P_{1}$\\
$x_{\varepsilon+k+1}\leftarrow x_{0}$
\end{mylisting}

Copiamo l'input:
\begin{mylisting}
$x_{1}\leftarrow x_{\varepsilon+1}$\\
$\vdots$\\
$x_{k}\leftarrow x_{\varepsilon+k}$
\end{mylisting}

Applichiamo $P_{2}$ e copiamo l'output in $x_{\varepsilon+k+2}$;
ripetiamo questa operazione per tutti gli altri programmi fino ad
applicare il programma $P_{n}$ e copiare il suo output in
$x_{\varepsilon+k+n}$.

A questo punto tutti gli input per P si trovano in
$x_{\varepsilon+k+1},\dots,x_{\varepsilon+k+n}$.  Copiamo questi
valori in $x_{1},\dots,x_{k}$ e applichiamo il programma P:

\begin{mylisting}
$x_{1}\leftarrow x_{\varepsilon+k+1}$\\
$\vdots$\\
$x_{k}\leftarrow x_{\varepsilon+k+n}$\\
$P$
\end{mylisting}

Dopo questa sequenza di operazioni il risultato della composizione si
trova in $x_{0}$.

\item La ricorsione: sia $F$ un programma che calcola $f$ e $G$ un
  programma per $g$. Per realizzare un programma che calcola $h$
  dobbiamo innanzitutto salvare le variabili di input e i risultati
  parziali di $f$ e $g$ in variabili non utilizzate dai due programmi
  $F$ e $G$.

  Sia $\rho(Q)=max\left\{ i|x_{i}\leftarrow\dots \in Q\right\}$ il
  massimo indice utilizzato nel programma Q; definiamo
  $\varepsilon=max\left\{ \rho(F),\rho(G),k+2\right\}$ il massimo
  indice utilizzato per entrambi i programmi.

Allora, prima copiamo l'input e mettiamo 0 in $x_{\varepsilon+k+2}$
per cominziare la ricorsione (per calcolare $h(x,0)$). Adesso possiamo
applicare il programma $F$.

Poi dobbiamo guardare se l'$y$ che ci è stato dato all'inizio è uguale
a zero; se è cosi, il programma deve terminare. Se non è cosi, il
programma deve calcolare $h(x,s(y))$ (la prima volta $s(y)$ sarà
1). Il programma G utilizza come input $x_{1},\dots,x_{k}$ per
$x_{1},\dots,x_{k}$, $x_{k+1}$ per $y$ e $x_{k+2}$ per il risultato
parziale $h(x,y)$. Quindi dobbiamo mettere l'input iniziale in
$x_{1},\dots,x_{k}$, il risultato al passo precedente che si trova in
$x_{\varepsilon+k+2}$ dobbiamo copiarlo in $x_{k+1}$ e, dopo aver
eseguito $G$, dobbiamo copiare il suo output da $x_{0}$ in $x_{k+2}$.
Infine calcoliamo il successore di $y$ e controlliamo se equivale
all'$y$ iniziale: in caso affermativo dobbiamo terminare, altrimenti
torniamo ad eseguire il ciclo. Allora, il programa sarà:

\begin{mylisting}
$x_{\varepsilon+1}\leftarrow x_{1}$\\
$\vdots$\\
$x_{\varepsilon+k}\leftarrow x_{k}$\\
$x_{\varepsilon+k+1}\leftarrow x_{k+3}$\\
$F$\\
$x_{\varepsilon+k+2}\leftarrow x_{0}$\\
$x_{\varepsilon+k+1}\leftarrow 0$\\
$loop:\; $if$\; x_{\varepsilon+k+1}=x_{\varepsilon+k+3}\; $then goto$\;stop$\\
$x_{\varepsilon+k+1}\leftarrow x_{\varepsilon+k+1}+1$\\
$x_{1}\leftarrow x_{\varepsilon+1}$\\
$\vdots$\\
$x_{k}\leftarrow x_{\varepsilon+k}$\\
$x_{k+1}\leftarrow x_{\varepsilon+k+1}$\\
$x_{k+2} \leftarrow x_{\varepsilon+k+2}$\\
$G$\\
$x_{\varepsilon+k+2} \leftarrow x_0$\\
goto$\; loop$\\
$stop:$
\end{mylisting}
\end{enumerate}
\end{proof}

% Controllare
\subsection{Esempi di funzioni primitive ricorsive}

\begin{esempio}[somma]
$h:\mathbb{N}^2 \to \mathbb{N}$, $h(x,y)=x+y$.
Possiamo usare lo schema semplificato
$\left\{ \begin{array}{ll}
	x+0=x\\
	x+s(y)=s(x+y)
\end{array}\right.$.\\
Nel nostro schema di ricorsione la stessa funzione si ottiene con $h$
tale:\newline
$$\begin{array}{ll}
	h(x,0)=f(x)=p_1^1(x)=x\\
	h(x,s(y))=g(x,y,h(x,y))=s(p_3^3(x,y,h(x,y)))=s(h(x,y)).
\end{array}$$\newline
\end{esempio}
%
\begin{esempio}[prodotto]

 $h:\mathbb{N}^2 \to \mathbb{N}$, $h(x,y)=x \cdot y$ che possiamo scrivere
come
$\left\{ \begin{array}{ll}
	x \cdot 0=0\\
	x \cdot s(y)= x \cdot y + x
\end{array}\right.$ e quindi: \newline
$$\begin{array}{ll}
	h(x,0)=f(x)=z(x)\\
	h(x,s(y))=g(x,y,h(x,y))=p_3^1(x,y,h(x,y))+p_3^3(x,y,h(x,y)).
\end{array}$$\newline
\end{esempio}
%
\begin{esempio}[fattoriale] Per adattare la definizione della funzione
fattoriale
$h:\mathbb{N} \to \mathbb{N}$, $h(x)=x!$ allo schema generale si considera
$\left\{ \begin{array}{ll}
	0!=1\\
	s(y)! = y! \cdot s(y)
\end{array}\right.$ e quindi si pu\`o prendere $h$ come segue: \newline
$$\begin{array}{ll}
	h(x,0)=f(x)=s(z(x))=1\\
	h(x,s(y))=g(x,y,h(x,y))= s(p_3^2(x,y,h(x,y)))) \cdot p_3^3(x,y,h(x,y))=
\\
	\hspace{1.8cm} = s(y)\cdot h(x,y).
	\end{array}$$\newline
\end{esempio}
%
%
\begin{esempio}[elevamento a potenza] $h:\mathbb{N}^2 \to \mathbb{N}$,
$h(x,y)=x^y$:
$\left\{ \begin{array}{ll}
	x^0=1\\
	x^{s(y)} = x^y \cdot x
\end{array}\right.$ e quindi: \newline
$$\begin{array}{ll}
	h(x,0)=f(x)=s(z(x))=1\\
	h(x,s(y))=g(x,y,h(x,y))= p_3^1(x,y,h(x,y)) \cdot p_3^3(x,y,h(x,y)) =
x\cdot h(x,y).
	\end{array}$$\newline
\end{esempio}
%
%
\begin{esempio}[predecessore] $p:\mathbb{N} \to \mathbb{N}$ tale che
$\left\{ \begin{array}{ll}
	p(0)=0\\
	p(s(y)) = y
\end{array}\right.$ e quindi: \newline
$$\begin{array}{ll}
	p(0)=f=0\\
	p(s(y))=g(y,p(y))= p_2^1(y,p(y)).
	\end{array}$$\newline
\end{esempio}
%
%
\begin{esempio}[sottrazione] $h:\mathbb{N}^2 \to \mathbb{N}$,
$ h(x,y) = \left\{ \begin{array}{ll}
	x \stackrel{\centerdot}{-} y \ se \ y \leq x\\
	0 \ altrimenti
\end{array}\right.$ dunque: \newline
$$\begin{array}{ll}
	h(x,0)=f(x)=x\\
	h(x,s(y))=p(h(x,y)).
	\end{array}$$\newline
\end{esempio}
%
%
\begin{esempio}[segno] $sgn:\mathbb{N} \to \left\{0,1\right\}$,
$ sgn(y) = \left\{ \begin{array}{ll}
	0 \ se \ y = 0\\
	1 \ se \ y > 0
\end{array}\right.$ che si pu\`o scrivere come:
 \[ sgn(y) = y \stackrel{\centerdot}{-} p(y) \]
che \`e primitiva ricorsiva per quanto visto negli esempi precedenti.
\end{esempio}
%
%
\begin{esempio}[controsegno] $\overline{sgn}:\mathbb{N} \to
\left\{0,1\right\}$,
$ \overline{sgn}(y) = \left\{ \begin{array}{ll}
	1 \ se \ y = 0\\
	0 \ se \ y > 0
\end{array}\right.$ che si pu\`o pensare come:
 \[ \overline{sgn}(y) = 1 \stackrel{\centerdot}{-} sgn(y) \]
dunque \`e primitiva ricorsiva.
\end{esempio}
%
%
\begin{esempio}[] $f(\overrightarrow{x},y)= \sum_{i=0}^{y} g(\overrightarrow{x},
i)$ con $g$ primitiva ricorsiva, allora:\\
 $ f(\overrightarrow{x}, 0) = g(\overrightarrow{x}, 0)  $ \\
 $ f(\overrightarrow{x}, s(y)) = f(\overrightarrow{x}, y) +
g(\overrightarrow{x}, s(y)). $
\end{esempio}
%
%
\begin{esempio}[] $f(\overrightarrow{x},y)= \prod_{i=0}^{y}
g(\overrightarrow{x}, i)$ con $g$ primitiva ricorsiva, allora:\\
$	f(\overrightarrow{x}, 0) = g(\overrightarrow{x}, 0) $ \\
$ 	f(\overrightarrow{x}, s(y)) = f(\overrightarrow{x}, y) \cdot
g(\overrightarrow{x}, s(y)). $
\end{esempio}
%
%
\begin{esempio}[] $ \chi_{\geq}(x,y) = \left\{ \begin{array}{ll}
	1 \ se \ x \geq y\\
	0 \ altrimenti
\end{array}\right.$
\[ \chi_{\geq}(x,y) = sgn (s(x) \stackrel{\centerdot}{-} y) . \]
\end{esempio}
%
%
\begin{esempio}[valore assoluto] Basta scriverlo come
\[ \left|x - y \right| = (x \stackrel{\centerdot}{-} y) + ( y
\stackrel{\centerdot}{-} x) \]
oppure
\[ \left|x - y \right| = (x \stackrel{\centerdot}{-} y) \chi_{\geq}(x,y)  +
( y \stackrel{\centerdot}{-} x) (1 - \chi_{\geq}(x,y)) .\]
\end{esempio}
%
%
\begin{esempio}[] $f(\overrightarrow{x}) = \left\{ \begin{array}{ll}
	g_1(\overrightarrow{x})\ se\ vale\  R(\overrightarrow{x})\\
	g_2(\overrightarrow{x})\ altrimenti
\end{array}\right.$ con \\$\chi_R(\overrightarrow{x}) = \left\{
\begin{array}{ll}
	1\ se\ R(\overrightarrow{x})\ vale\\
	0\ altrimenti
\end{array}\right.$ con $g_1(\overrightarrow{x})$, $g_2(\overrightarrow{x})$
e $\chi_R(\overrightarrow{x})$ primitive ricorsive.\\
Allora
\[ f(\overrightarrow{x}) = g_1(\overrightarrow{x}) \chi_R(\overrightarrow{x}) +
 g_2(\overrightarrow{x}) (1 - \chi_R(\overrightarrow{x})). \]
\end{esempio}
%

% Alessandro Onnivello
\section{Dalle funzioni primitive ricorsive alle funzioni ricorsive}

Nel paragrafo precedente abbiamo definito l'insieme delle funzioni
primitive ricorsive; possiamo dire che queste siano sufficienti per
rappresentare tutte le funzioni calcolabili da una macchina di Turing?
La risposta è negativa, come andremo ora a dimostrare.

\begin{proposizione}\label{PRsonoTotali}
Le funzioni primitive ricorsive sono totali.
\end{proposizione}
\begin{proof}
Si dimostra per induzione sulla struttura delle funzioni primitive
ricorsive.  Poichè le funzioni di base sono funzioni totali dobbiamo
verificare che la composizione e la ricorsione primitiva preservano la
totalità delle funzioni composte.

Per la composizione dobbiamo verificare che la funzione composta
$f(g_1, \dots, g_k)$ è totale. Per ipotesi induttiva
(strutturale) $f,g_1,\dots,g_k$ sono totali. Per la totalità di
$g_1,\dots,g_k$ tutti gli argomenti di $f$ sono definiti e, poichè $f$
è totale, anche la composizione è definita.

Per la ricorsione primitiva dobbiamo dimostrare che
$h:\mathbb{N}^{k+1} \to \mathbb{N}$ è definita $\forall y \in
\mathbb{N}$ e $\forall \vec{x} \in \mathbb{N}^k$.

$$h(\vec{x},y)=\left\{
\begin{array}{ll} h(\vec{x}, 0) = f(\vec{x})\\
                  h(\vec{x}, s(y)) = g(\vec{x},y,h(\vec{x},y))
\end{array} \right.$$

Si dimostra per induzione su $y$.

$y = 0$: $h(\vec{x}, 0) = f(\vec{x})$\\ $f$ è totale per ipotesi
induttiva quindi $\forall \vec{x} \in \mathbb{N}^k (\vec{x}, 0) \in
dom(h)$ come atteso

$y + 1$: $h(\vec{x}, y + 1) = g(\vec{x}, y, h(\vec{x}, y))$\\
$h(\vec{x}, y)$ è totale per ipotesi induttiva su $y$ quindi
$h(\vec{x}, y)$ è definita. Poichè $g$ totale per ipotesi induttiva e
$h(\vec{x}, y)$ è definita allora $\forall \vec{x} \in \mathbb{N}^k \;
(\vec{x}, y + 1) \in dom(h)$.
\end{proof}

Ora, per quanto detto nella precedente proposizione, l'insieme delle funzioni
primitive ricorsive permette di rappresentare solo funzioni calcolabili totali
ma, come abbiamo visto nel primo capitolo, le macchine di Turing possono
calcolare anche funzioni parziali. Quindi possiamo affermare che le
funzioni primitive ricorsive non rappresentano tutte le funzioni Turing
calcolabili.

Ma limitandoci al caso delle funzioni totali, possiamo dire che queste
siano tutte primitive ricorsive? Anche in questo caso la risposta è
negativa come vedremo nel prossimo paragrafo.

\section{Metodo diagonale di Cantor}
Per poter dimostrare che l'insieme delle funzioni primitive ricorsive
non contiene tutte le funzioni totali calcolabili andremo ora a
introdurre il metodo diagonale (o diagonalizzazione) di Cantor.  Il
metodo di diagonalizzazione è una tecnica dimostrativa, ideata da
Georg Cantor per provare la non numerabilità dei numeri reali, che
risulta molto utile anche nell'ambito della logica matematica e della
teoria della computabilità come vedremo in seguito.

Il metodo diagonale di Cantor consiste nel costruire una funzione $\chi$
che differisce da un'insieme infinito di funzioni $\chi_0,\chi_1,\dots$
effettivamente enumerabile. Si costruisce quindi la seguente tabella
dove ogni colonna contiene una funzione unaria e le righe contengono
la sequenza dei naturali, ovvero tutti i possibili argomenti.
\begin{table}[!h]
\begin{center}
\begin{tabular}{|c|c|c|c|c}

\hline
 & $\chi_0$ & $\chi_1$ & $\chi_2$ & $\ldots$\\
\hline
0 & $\chi_0(0)$ & $\chi_1(0)$  & $\chi_2(0)$  & $\ldots$\\
\hline
1 & $\chi_0(1)$ & $\chi_1(1)$  & $\chi_2(1)$  & $\ldots$\\
\hline
2 &  $\chi_0(2)$ & $\chi_1(2)$  & $\chi_2(2)$  & $\ldots$\\
\hline
$\vdots$ & $\vdots$ &       $\vdots$  & $\vdots$  & $\ddots$\\

\end{tabular}
\end{center}
\caption{Metodo diagonale di Cantor}
\end{table}

A questo punto possiamo costruire una funzione $\chi$ in modo tale che
differisca da ogni funzione enumerata nella tabella; vogliamo quindi
costruire una funzione tale che $\forall i \; \chi(i) \neq
\chi_{i}(i)$, ovvero che differisce da ogni altra funzione elencata
nella tabella almeno sulla diagonale. Questa nuova funzione, creata
tramite una procedura effettiva, non può appartenere all'insieme di
partenza, poichè altrimenti differirebbe da sè stessa sulla diagonale.

\begin{proposizione}\label{TotCalcNonNum}
Data una qualunque lista \emph{effettiva} di tutte le funzioni totali
calcolabili, unarie per semplicit\`a, $f_{0},\; f_{1},\; f_{2},
\cdots$ esiste sempre una funzione $g$ totale calcolabile che non
compare nella lista.
\end{proposizione}
\begin{proof}
Sia $f_{0},\; f_{1},\; f_{2}, \cdots$ una lista effettiva di funzioni
totali calcolabili, allora costruiamo la funzione $g$ nel seguente
modo:

\begin{center}
$g: \mathbb{N} \to \mathbb{N}$\\
$g(x) = f_{x}(x) + 1$
\end{center}
\begin{table}[!h]
\begin{center}
\begin{tabular}{|c|c|c|c|c}

\hline
 & $f_0$ & $f_1$ & $f_2$ & $\ldots$\\
\hline
0 & $f_0(0) \mathbf{+1}$ & $f_1(0)$  & $f_2(0)$  & $\ldots$\\
\hline
1 & $f_0(1)$ & $f_1(1) \mathbf{+1}$  & $f_2(1)$  & $\ldots$\\
\hline
2 &  $f_0(2)$ & $f_1(2)$  & $f_2(2) \mathbf{+1}$  & $\ldots$\\
\hline
$\vdots$ & $\vdots$ &       $\vdots$  & $\vdots$  & $\ddots$\\
\end{tabular}
\end{center}
\caption{Funzione di diagonalizzazione $\mathbf{g}$}
\label{diagG}
\end{table}

$g$ è sicuramente totale calcolabile poichè, essendo la lista di
funzioni effettiva è quindi possibile, dato un qualsiasi $n \in
\mathbb{N}$, calcolare l'ennesima funzione, valutarla in $n$ ed
aggiungerci 1. È chiaro che $g$ non può appartenere alla lista, poichè
presa una qualsiasi colonna $n$ della tabella~\ref{diagG} che
rappresenta l'immagine dell'ennesima funzione, alla riga ennesima
$g(n) \neq f_{n}(n)$ poichè per definizione $g(n)=f_{n}(n)+1$. Perciò
non esiste alcun $n$ tale che $f_{n}(x)=g(x)$ per ogni $x \in
\mathbb{N}$ e quindi $g$ non è contenuta nella nostra lista.
\end{proof}

Fatto questo ragionamento, ne consegue che non è possibile enumerare
in modo effettivo tutte le funzioni totali calcolabili.

Possiamo usare ora la proposizone~\ref{TotCalcNonNum} per dimostrare il seguente

\begin{teorema}\label{diagRic} Esiste una funzione totale calcolabile che non
è primitiva ricorsiva.
\end{teorema}

\begin{proof} Dobbiamo innanzitutto costruire una lista \emph{effettiva} di
tutte le funzioni primitive ricorsive (per semplicità, supponiamole ad
un argomento).  Un modo semplice per farlo è considerare l'insieme
delle funzioni primitive ricorsive $\mathcal{PR}$ ed un formalismo,
ovvero un insieme finito di simboli con il quale sia possibile
descrivere tutte le possibili derivazioni associate alle funzioni di
$\mathcal{PR}$. Un siffatto alfabeto può essere ad esempio:
$$\Sigma = \lbrace a, b, c, \cdots ,z , 0, 1, 2, \cdots , 9, (, ) \rbrace$$
Ogni derivazione associata ad una funzione in $\mathcal{PR}$ \`e quindi
descrivibile con una opportuna sequenza finita di simboli $\Sigma^{*}$. \`E
inoltre facile definire una procedura che verifichi se una data stringa $\sigma
\in \Sigma^{*}$ corrisponde a una derivazione legittima per una funzione in
$\mathcal{PR}$. \`E pertanto possibile enumerare tutte le derivazioni legittime
per funzioni primitive ricorsive esaminando le stringhe in $\Sigma^{*}$ ordine
de lunghezza e a parit\`a di lunghezza ordinandole in modo alfabetico.

Abbiamo cos\`i ottenuto una enumerazione effettiva per tutte le funzioni
dell'insieme $\mathcal{PR}$.

Ma per quanto detto nella proposizone~\ref{TotCalcNonNum} sappiamo che le
funzioni totali non sono enumerabili effettivamente quindi questo implica che
esiste una funzione totale calcolabile che non appartiene l'enumerazione che
abbiamo appena definito e che quindi non appartiene all'insieme $\mathcal{PR}$.
\end{proof}

A questo punto ci chiediamo: cosa ci manca per avere tutte le funzioni
calcolabili?
Ripensando a quanto visto fino ad ora, il problema pu\`o stare o nel pretendere
di avere una lista effettiva delle funzioni calcolabili (cosa che per\`o \`e
ragionevole a farsi: lo abbiamo visto poco fa), o nell'imporre la propriet\`a di
totalit\`a alle funzioni. Dunque vediamo se lasciando cadere questa assunzione
riusciamo a classificare tutte le funzioni calcolabili.


\section{Le funzioni ricorsive}
Avendo osservato che le funzioni primitive ricorsive non
esauriscono tutte le funzioni calcolabili, il passo successivo \`e  quello di
estenderle ulteriormente. In particolare ci sar\`a bisogno di trovare
anche funzioni parziali nella nostra definizione di ``ricorsivit\`a''. Perci\'o
introduciamo
la seguente funzione: \\

f. \textbf{minimizzazione}: sia $f: \mathbb{N}^{d+1} \to \mathbb{N}$ (anche
parziale)\\
definiamo la funzione di minimizzazione $h: \mathbb{N}^{d} \to \mathbb{N}$ come
$$ h(\vec{x}):= \mu_{y}(f(\vec{x},y)) $$
dove

\[\mu_{y}(f(\vec{x},y)) = 
\begin{cases}
il \; minimo \; y \; tale \; che: \\
\qquad (i) \; f(\vec{x},z)\downarrow \; \forall z \leq y \; e \\
\qquad (ii) \; f(\vec{x},y)=0 & \text{se y esiste,} \\
non \; definita \; & \text{se $\exists z < y \; f(\vec{x},z) \uparrow$}\\
 & \text{o se $f(\vec{x},y)\neq 0 \; \forall y \in \mathbb{N} $}
\end{cases} \]

con $\mu$ che prende il nome di operatore di minimizzazione.
\begin{teorema} Le funzioni ricorsive sono b-programma-calcolabili.
\end{teorema}
\begin{proof} Poich\`e abbiamo gi\`a dimostrato che le tre funzioni di base
(funzione zero, successore e proiezioni), la composizione generalizzata e la
ricorsione primitiva sono calcolabili da un b-programma ci resta solo da far
vedere che questo vale anche per la minimizzazione. Infatti se
$f:\mathbb{N}^{d+1} \to \mathbb{N}$ \`e computabile con un b-programma $\alpha$
allora possiamo eseguire la seguente assegnazione:
$$x_i \leftarrow f(\vec{x_{j}}, x_k)$$
(dove con $\vec{x_{j}}$ vogliamo rappresentare in modo compatto una lista di d
variabili di input)che significa "`dai come input $\vec{x_{j}}$ e $x_k$ al
b-programma $\alpha$ che calcola $f(\vec{x_{j}}, x_k)$ e dai output risultante
come valore a $x_i$"'. Dunque il b-programma che calcola la funzione di
minimizzazione su $\vec{x}$ \`e il seguente:\\
   \begin{mylisting}
       $x_0 \leftarrow 0$ \\
       $loop$: $\vec{x_{1}} \leftarrow \vec{x}$\\
       $x_{d+2} \leftarrow f(\vec{x_{1}}, x_0)$\\
       if $x_{d+2} = 0$ then goto $stop$\\
       $x_0 \leftarrow x_0 + 1$\\
       goto $loop$
  \end{mylisting}
\end{proof}

Una volta dimostrato che la minimizzazione è calcolabile dai
b-programmi possiamo dare la seguente
\begin{definizione} Una funzione $f:\mathbb{N}^{d} \to \mathbb{N}$ (totale o
parziale) si dice \emph{ricorsiva} se si ottiene dalle funzioni
iniziali (a.--c.) applicando la composizione, la ricorsione e la
minimizzazione (d.--f.).
\end{definizione}

Una cosa importante da notare \`e che l'utilizzo dell'operatore $\mu$ consente
di ottenere funzioni parziali anche a partire funzioni totali, come vedremo nei
seguenti esempi.
\begin{esempio} data $f:\mathbb{N} \to \mathbb{N}$ definita nel seguente modo:
$$f(x)= \left\{ \begin{array}{ll}
il \; min \; y \; t. \, c. \; x+y=0 & se \; \exists \; y\\
non \; definita & se \; x+y \neq 0 \; \forall y \in \mathbb{N}\\
\end{array} \right.$$
utilizzando la minimizzazione pu\`o essere scritta nel seguente modo:
$$f(x)= \mu_{y}(x+y)$$

Si noti che, essendo $x \in \mathbb{N}$, se $x \neq 0 \; f$ non \`e definita
anche se la funzione somma su cui viene applicato l'operatore di minimizzazione
\`e totale.
\end{esempio}

\begin{esempio} data $f:\mathbb{N}^{2} \to \mathbb{N}$ definita nel seguente
modo:
$$f(x,y)= \left\{ \begin{array}{ll}
x/y & se \; y \vert x\\
non \; definita & se \; y \nmid x\\
\end{array} \right.$$
pu\`o essere scritta nel seguente modo:
$$f(x,y) = \mu_{z}(\vert (z*y) - x \vert) $$
anche in questo caso la funzione $g(x,y,z)=\vert (z*y) - x \vert$ \`e totale, ma
combinata con l'operatore di minimizzazione consente di ottenere la funzione
parziale $f$.
\end{esempio}

\section{Relazioni ricorsive}
\begin{definizione} $R \subseteq \mathbb{N}^{n}$ è una \emph{relazione
ricorsiva} se esiste una funzione ricorsiva $\chi_R$ che assume solo valori 0 e
1 e che soddisfa
$$\chi_R(x_1, \ldots, x_n):= \left \{ \begin{array}{ll}
                                      1 & (x_{1}, x_{2}, \cdots, x_{n}) \in R\\
                                      0 & (x_{1}, x_{2}, \cdots, x_{n}) \not \in
 R\\
                                      \end{array} \right. $$
\end{definizione}
\begin{proposizione} Se $R, S \subseteq \mathbb{N}^{n}$ sono relazioni
ricorsive allora anche $R\land S$, $R \vee S$, $\neg R$ sono relazioni
ricorsive. Inoltre, se $w \in \mathbb{N}$, anche $\forall y \leq
w. R(x_1, \ldots, x_{n-1}, y)$ e $\exists y \leq w. R(x_1, \ldots,
x_{n-1}, y)$ sono relazioni ricorsive.
\end{proposizione}
\begin{proof} Si pone
\begin{itemize}
\item $\chi_{R\land S} = \chi_R \chi_S$;
\item $\chi_{R\vee S} = \chi_R + \chi_S - \chi_R  \chi_S$;
\item $\chi_{\neg R} = 1 - \chi_R$;
\item $\chi_{\forall y
\leq w. R(x_1, \ldots, x_{n-1}, y)} = \prod_{i=0}^{w} \chi_R(x_1, \ldots,
x_{n-1},i)$;
\item $\chi_{\exists y
\leq w. R(x_1, \ldots, x_{n-1}, y)} = sgn(\sum_{i=0}^{w} \chi_R(x_1, \ldots,
x_{n-1},i))$
\end{itemize}
\end{proof}

Grazie alla definizione data sopra, ci \`e possibile giustificare la
\emph{definizione per casi} delle funzioni ricorsive.

Supponiamo di avere una funzione $f: \mathbb{N}^{n} \to n$ t. c.
$$f(\vec{x}):= \left \{ \begin{array}{ll}
                    g_1(\vec{x}) & $se $ \vec{x} \in R_{1}\\
                    \vdots \\
                   g_k(\vec{x}) & $se $ \vec{x} \in R_{k}\\
                    \end{array} \right.$$
con $g_{1}, g_{2}, \cdots, g_{k}:\mathbb{N}^{n} \to \mathbb{N}$ funzioni
ricorsive e dove $R_{1}, R_{2}, \cdots, R_{k} \subseteq \mathbb{N}^{n}$ sono
relazioni ricorsive mutualmente esclusive ed esaustive, ossia abbiamo che:
\begin{itemize}
\item $\sum_{i=1}^{k} \chi_{R_{i}}(\vec{x})=1$ e che
\item se $R_{i}(\vec{x})$ vale allora $\forall j \neq i \; \chi_{R_{i}}(\vec{x})
\cdot \chi_{R_{j}}(\vec{x})=0$.
\end{itemize} 
Possiamo quindi prendere le funzioni caratteristiche $\chi_{R_{1}},
\chi_{R_{2}}, \cdots, \chi_{R_{k}}$ e definire la funzione $f$ come:
$$f(\vec{x})= \sum_{i=1}^{k} g_i(\vec{x})\chi_{R_{i}}(\vec{x})$$ e
quindi essendo $f$ ottenuta per composizione di funzioni ricorsive \`e
anch'essa ricorsiva.  Quindi se vogliamo descrivere una funzione che
assume valori diversi in casi diversi lo possiamo fare mantenendoci
all'interno delle funzioni primitive ricorsive. Questo può permetterci
di definire funzioni come quella del prossimo esempio.

\begin{esempio} Data la seguente funzione:
\[ f: \mathbb{N}^{2} \to \mathbb{N} \]
\[ f(x,y)= \begin{cases}
x-y	& \text{se $x \geq y$}\\
x+y & \text{se $x<y$}
\end{cases}\]
la sua definizione per casi \`e la seguente:\\
$f(x,y) = \sum_{i=1}^{2} g_{i} \cdot \chi_{R_{\geq}}(x,y) = (x-y)\cdot
\chi_{R_{\geq}}(x,y) + (x+y)\cdot \neg \chi_{R_{\geq}}(x,y) $
Si lascia come esercizio al lettore il trovare $\chi_{R_{\geq}}$.
\end{esempio}

% Alessandro Bruni

\section{Equivalenze}

Per riassumere il percorso seguito finora presentiamo il seguente schema.
Ad ogni passaggio abbiamo introdotto un linguaggio pi\`u
raffinato senza per\`o aggiungere nuove funzioni:
 ogni volta abbiamo aggiunto qualcosa che fosse definibile in
termini del precedente e quindi in particolare di macchine di Turing.
\begin{align*}
{\rm Macchine\ }&{\rm di\ Turing\ }\\
\bigcup &|\\
{\rm Macchine\ }&{\rm a\ registri\ }\\
\bigcup &|\\
{\rm Progra}&{\rm mmi}\\
\bigcup &|\\
{\rm Funzioni\ }&{\rm ricorsive\ }\\
\bigcup &| \; {\rm?}\\
{\rm Macchine\ }&{\rm di\ Turing\ }\\
\end{align*}

È valida valida l'ultima inclusione? in caso affermativo tutti i
concetti di funzioni calcolabili descritti fino ad ora sono tra loro
equivalenti. Il seguente teorema dimostra che quest'inclusione è
effettivamente valida.

\begin{teorema}[Le funzioni Turing-calcolabili sono ricorsive]
Ovvero, per ogni MdT possiamo costruire una funzione ricorsiva che la calcola.
\end{teorema}

\begin{proof}
Dobbiamo innanzitutto trovare un modo per rappresentare il nastro (ovvero la
memoria) della MdT. Per semplicità di esposizione poniamoci nel caso $\sigma =
\{ B, 1 \}$, poichè ogni altro alfabeto finito può avere una rappresentazione
simile nei naturali, scegliendo una rappresentazione in base $|\sigma|$
dell'input.

Una MdT computa una funzione $f: \mathbb{N}^k \rightarrow \mathbb{N}$,
possibilmente parziale, quindi lo stato del nastro all'inizio della
computazione è una sequenza di caratteri che contiene le variabili in
input:
$$\dots BB\overline{x_1+1}B\overline{x_2+1}B...B\overline{x_k+1}BB\dots $$
mentre alla fine del calcolo la macchina dovrebbe lasciare nel nastro
il risultato della funzione che calcola:
$$\dots BB\overline{f(x_1,\dots,x_k)+1}BB\dots $$

Diamo ora le seguenti definizioni, per meglio comprendere la nostra codifica
del nastro in $\mathbb{N}$.

\begin{definizione}[Numerale sinistro]
Si definisce numerale sinistro la sequenza di caratteri, letti da sinistra a
destra a partire dal primo 1, che si trovano strettamente a sinistra della
testina della MdT, dove 1 rappresenta la cifra 1 e $B$ rappresenta la cifra 0.
\end{definizione}
\begin{definizione}[Numero sinistro]
Si dice numero sinistro l'interpretazione nei naturali del numerale sinistro,
ovvero quel numero $n \in \mathbb{N}$ la cui rappresentazione in base 2 è
esattamente il numerale sinistro.
\end{definizione}
\begin{definizione}[Numerale destro]
È la sequenza di caratteri a destra della testina (compreso quello sopra la
testina stessa) lette da destra verso sinistra, a partire dal primo 1 che
compare sul nastro.
\end{definizione}
\begin{definizione}[Numero destro]
È l'interpretazione nei naturali del numerale destro.
\end{definizione}

È facile comprendere da queste definizioni che il nastro e la testina della
macchina di Turing possono essere rappresentati da una coppia di interi, ovvero
il numero sinistro e il numero destro. Inoltre il carattere esattamente sotto
la testina è 1 se il numero destro è dispari, 0 se pari.

\begin{esempio}
Per questa macchina di Turing
\begin{align*}
\dots BB11B&111B1111BB \dots\\
&\,\uparrow
\end{align*}
il numerale sinistro è 1101, il numero sinistro è 13; il numerale destro è
1111011, il numero destro è 123.
\end{esempio}

Ora dobbiamo poter simulare su questa coppia di numeri le operazioni che la
macchina di Turing svolge sul nastro. Queste sono:
\begin{enumerate}
 \item scrivere un 1 nella casella sulla quale sta la testina;
 \item scrivere uno 0 sulla casella;
 \item spostare la testina a destra (R);
 \item spostare la testina a sinistra (L);
\end{enumerate}

Per le prime due è facile trovare una funzione ricorsiva che modifica il
numero destro in modo da rappresentare la scrittura di un 1 o di uno 0: nel
primo caso basta sommare 1 se il numero è pari (ovvero nella posizione
sottostante alla testina della MdT simulata vi è uno 0), nel secondo caso è
sufficente sottrarre 1 se il numero è dispari.

Definiamo quindi le funzioni che svolgono queste due operazioni:
$$w_1(x) = \left\{
\begin{array}{ll}
x+1 & \text{se } 2 \mid x\\
x & \text{se } 2 \nmid x
\end{array}
\right.$$
che simula l'operazione di scrivere un 1, e
$$w_B(x) = \left\{
\begin{array}{ll}
x & \text{se } 2 \mid x\\
x-1 & \text{se } 2 \nmid x
\end{array}
\right.$$
che simula l'operazione di scrivere uno 0.

Per simulare lo spostamento della testina sul nastro bisogna osservare come
cambiano i numerali quando la testina si muove a destra o a sinistra. Siano $l$
e $r$ rispettivamente il numero sinistro e il numero destro prima dello
spostamento della testina, $l'$ ed $r'$ il numero sinistro ed il numero destro
dopo lo spostamento della testina.
\begin{enumerate}
 \item Se la testina si muove verso sinistra ed $l$ è pari allora vuol
   dire che a sinistra della testina c'è uno 0, che passa da cifra
   meno significativa del numerale sinistro ad essere la cifra meno
   significativa del numerale destro.  Quindi $l' = l/2$ ed $r' = r
   \cdot 2$.
 \item Se la testina si muove verso sinistra ed $l$ è pari allora vuol
   dire che a sinistra della testina c'è un 1, che passa dal numerale
   sinistro al numerale destro; quindi $l' = (l-1)/2$ ed $r' = r \cdot
   2 + 1$.
 \item Se la testina si muove verso destra e la cifra meno
   significativa del numerale destro è un 1, questa passa al numerale
   sinistro per cui: $l' = l \cdot 2 + 1$, $r' = (r-1)/2$.
 \item Se la testina si muove verso destra e la cifra meno
   significativa del numerale destro è uno 0, $l' = l \cdot 2$ e $r' =
   r/2$.
\end{enumerate}

Vediamo ora come codificare le quintuple $\langle q_i, S_i, S_j,
\{L,R\}, q_j \rangle$ che descrivono una macchina di Turing.

Queste definiscono una funzione di transizione $\tau : \mathbb{N}^2
\rightarrow \mathbb{N}^3$ e, poichè le nostre funzioni ricorsive sono
tutte della forma $f:\mathbb{N}^k \rightarrow \mathbb{N}$, dobbiamo
trovare una funzione di codifica delle ennuple $\nu: \mathbb{N}^n
\rightarrow \mathbb{N}$ ed una classe di funzioni di proiezione
$\nu_i: \mathbb{N} \rightarrow \mathbb{N}$ che estrapolano la
proiezione dell'i-esima componente dall'ennupla codificata nei
naturali.

Delle funzioni adatte a questo scopo sono le seguenti:
$$\nu(\vec{x}) = \prod_{i=1}^n p_i^{x_i}$$ per la codifica
e $$\nu_i(w) = \max_x.p_i^x \mid w$$ per la decodifica dell'i-esima
componente, dove $p_i$ indica l'i-esimo numero primo. Queste funzioni
sfruttano l'unicità della fattorizzazione in numeri primi dei
naturali.

La funzione di codifica è chiaramente primitiva ricorsiva, in quanto
composizione di funzioni primitive ricorsive, ovvero il prodotto e
l'elevamento a potenza.

Per verificare che anche la funzione di proiezione è primitiva
ricorsiva dobbiamo introdurre gli operatori di minimizzazione e
massimizzazione limitata, rispettivamente $\mu_{w<n}[f](\vec{x})$ e
$\max_{w<n}[f](\vec{x})$, che calcolano il minimo e il massimo $w \in
[0,n]$ tale per cui $f(\vec{x}, w) = 0$.

$$\mu_{w<n}[f](\vec{x}) = \sum_{i=0}^n sgn\left(\prod_{j=0}^i
f(\vec{x},w)\right)$$

Sia $h(\vec{x},y) = f(\vec{x},n-y)$, definiamo la funzione di
massimizzazione in questo modo:

$$\max_{w<n}[f](\vec{x}) = n-\mu_{w<n}[h](\vec{x})$$

Chiaramente sia la minimizzazione che la massimizzazione limitate sono
funzioni primitive ricorsive, in quanto definite in termini di altre
funzioni primitive ricorsive.

Ora definiamo la funzione caratteristica della divisione:

\begin{align*}
div(x,y) =& \left\{
\begin{array}{ll}
1 & \text{se } x \mid y\\
0 & \text{se } x \nmid y
\end{array}
\right. \\
=& \, sgn\left( \left| \mu_{w<y+1}.\left|x \cdot w - y \right| - (y+1)\right|
\right)
\end{align*}

Infine possiamo definire la funzione di proiezione nel seguente modo:
$$\nu_i(w) = \max_{x<w}.\left| div(p_i^x, w) - 1\right|$$ e quindi
abbiamo dimostrato che anche la massimizzazione è primitiva rcorsiva.

Con gli strumenti che ci siamo appena procurati possiamo ora costruire
la codifica della funzione di transizione. Intanto ci serve una mappa
dei simboli che usiamo per descrivere la macchina di Turing nei
naturali:
\begin{align*}
q_i &\rightsquigarrow i+1\\
B &\rightsquigarrow 0\\
1 &\rightsquigarrow 1\\
L &\rightsquigarrow 1\\
R &\rightsquigarrow 2\\
\end{align*}

La funzione di transizione, dato un insieme $S$ di regole per la
macchina di Turing, è la seguente:
$$\tau(x,y) = \left\{
\begin{array}{ll}
\nu(u,v,w) & \text{se $\langle q_i, S_i, S_j, D, q_j \rangle \in
  S$}\\ & \text{e $q_i \rightsquigarrow x$, $S_i \rightsquigarrow y$,
  $S_j \rightsquigarrow u$, $D \rightsquigarrow v$, $q_j
  \rightsquigarrow w$}\\ \nu(0,0,0) & \text{se non si applica nessuna
  regola}
\end{array}
\right.$$ è una funzione definita per casi su un insieme finito di
regole $S$, per cui, come dimostrato nei paragrafi precedenti, è
primitiva ricorsiva.

Codifichiamo ora la funzione che esegue sul nastro la tripla ottenuta
dalla funzione di transizione. Siano $left(l,r)$ e $right(l,r)$ le
funzioni che simulano lo spostamento della testina rispettivamente a
sinistra ed a destra secondo le regole date in precedenza. Un passo
della MdT è codificato dalla seguente funzione:

$$step(l,r,x) = \left\{
\begin{array}{ll}
left(l, w_0(r)) \cdot p_3^{\nu_3(x)} & \text{se }\nu_1(x) = 0, \nu_2(x) = 1\\
left(l, w_1(r)) \cdot p_3^{\nu_3(x)} & \text{se }\nu_1(x) = 1, \nu_2(x) = 1\\
right(l, w_0(r)) \cdot p_3^{\nu_3(x)} & \text{se }\nu_1(x) = 0, \nu_2(x) = 2\\
right(l, w_1(r)) \cdot p_3^{\nu_3(x)} & \text{se }\nu_1(x) = 1, \nu_2(x) = 2\\
\end{array}
\right.$$

Questa funzione, definita per ricorsione primitiva, ci da lo stato
della macchina dopo $t$ passi:
\begin{align*}
f(l,r,0) &= \nu(l,r,1)\\
f(l,r,t+1) &= step(\nu_1(n), \nu_2(n), \tau(\nu_3(n), \nu_2(n) \mod 2) )
\end{align*}
con $n = f(l,r,t)$.

Vediamo come codificare una k-tupla di una funzione ricorsiva nel valore $r$.
Intanto codifichiamo un singolo elemento:
\begin{align*}
g(r,0) &= 2\cdot r + 1\\ 
g(r,x+1) &= 2\cdot f(r, x) + 1\\
\end{align*}
usando questa funzione codifichiamo la tupla:
$$cod(x_1, \dots, x_n) = g(\dots 2\cdot g(0, x_n), x_1)$$
e decodifichiamo il risultato:
$$dec(r) = \mu_x.r+1\dot{-}2^x$$

Ora non ci resta che costruire la funzione che esegue la macchina di
Turing sull'input e che ritorna la codifica del nastro alla
terminazione: $$MdT(\vec{x}) = dec(\nu_2(f(0,cod(\vec{x}),\mu_t .
|\nu_3(f(0,cod(\vec{x}),t)) = 0|)))$$

In questo modo abbiamo definito una funzione ricorsiva che svolge
l'intera computazione di una macchina di Turing, se questa termina, ed
è indefinita altrimenti. In effetti si può dire qualcosa in più: fino
alla fine della nostra costruzione abbiamo usato solo funzioni
primitive ricorsive, quindi totali per la
Proposizione~\ref{PRsonoTotali}. Ovvero è sempre possibile sapere qual
è lo stato della computazione della macchina di Turing (rappresentato
dai numerali sinistro e destro) ad un dato tempo $t$, la vera
incognita sta nella terminazione della macchina, com'era giusto
aspettarsi.

È interessante inoltre notare che la costruzione di questa funzione è
quasi del tutto indipendente dalla macchina di Turing che andiamo a
simulare con la nostra funzione ricorsiva, se non per la funzione di
transizione $\tau$ che non è altro che la rappresentazione
dell'insieme di regole date in pasto alla macchina. Potremmo dunque
benissimo pensare sostituire $\tau$ con un'altra funzione, più
sofisticata, che prende in input le quintuple codificate, lo stato
iniziale e il carattere letto e ritorna la tripla che codifica il
carattere da scrivere, la direzione in cui spostare la testina e lo
stato finale; questa funzione è in grado di prendere una qualsiasi
macchina di Turing codificata ed eseguirla su un qualsiasi
input. Inserendo questa particolare funzione di transizione e
sostituendola nel nostra prova otteniamo la macchina di Turing
universale, in grado cioè di eseguire una qualunque macchina di
Turing.

Abbiamo quindi dimostrato che per ogni macchina di Turing esiste una
funzione ricorsiva che la computa ed abbiamo visto come costruirla.
\end{proof}

\subsection{Tesi di Church-Turing}
Poichè tutti questi (ed altri) tentativi di definire tutto ciò che è
effettivamente calcolabile portano alla stessa classe di funzioni, si
può pensare che la nostra nozione intuitiva di calcolabilità coincida
esattamente con ognuna di queste definizioni.
